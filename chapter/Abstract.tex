\begingroup
\let\titlepage\par
\let\endtitlepage
\let
\selectlanguage{ngerman}
\begin{abstract}
Im Berliner Verkehrswesen ist ein deutlicher Trend zu bemerken. Das Fahrrad wird zum ökologischen und gesundheitlichen, aktiven Lebensstil und wird dem hohen Verkehrsaufkommen der Automobile, insbesondere in der Stadtregion, entgegenwirken. “Fahrradfahren boomt in Berlin stärker als bislang bekannt”  (J.Anker, Berliner Morgenpost, am 6.06.2014)\\
Neue Fahrradwege und Vergrößerung des Fahrradstraßennetzes sind regionale Baumaßnahmen, die dabei aktuell diesen Fahrradtrend unterstützen. Grund der neuen Fahrradeuphorie ist nicht zuletzt die erfolgreiche Etablierung der E-Bikes. E-Bikes erfreuen sich großer Beliebtheit und ermöglichen auch längere Touren ohne große Anstrengung.
\end{abstract}
\selectlanguage{english}
\begin{abstract}
Im Berliner Verkehrswesen ist ein deutlicher Trend zu bemerken. Das Fahrrad wird zum ökologischen und gesundheitlichen, aktiven Lebensstil und wird dem hohen Verkehrsaufkommen der Automobile, insbesondere in der Stadtregion, entgegenwirken. “Fahrradfahren boomt in Berlin stärker als bislang bekannt”  (J.Anker, Berliner Morgenpost, am 6.06.2014)\\
Neue Fahrradwege und Vergrößerung des Fahrradstraßennetzes sind regionale Baumaßnahmen, die dabei aktuell diesen Fahrradtrend unterstützen. \\
Grund der neuen Fahrradeuphorie ist nicht zuletzt die erfolgreiche Etablierung der E-Bikes. E-Bikes erfreuen sich großer Beliebtheit und ermöglichen auch längere Touren ohne große Anstrengung.
\end{abstract}
\endgroup
