\begingroup
\let\titlepage\par
\let\endtitlepage
\let
\selectlanguage{ngerman}
\begin{abstract}
In dieser Arbeit wird eine Anwendung entwickelt, die zur Unterstützung der Fahrt mit dem Fahrrad Ampelphasen vorhersagt und eine Empfehlung ausspricht. Sofern diese eingehalten wird, soll ein Passieren der grünen Ampelphasen ohne Anhalten ermöglicht werden.\\
Hierzu wird analysiert welche Studien, Projekte oder Anwendungen es zu diesem Thema bereits gibt. Es wird diskutiert, ob sich für die Realisierung eines Prototyps eine mobile Anwendung oder eher eine \gls{Arduino}installation anbietet. Basierend auf den beiden möglichen Entscheidungswegen werden die technischen und physikalischen Grundlagen erklärt. Hier werden Definitionen und Entwicklungswerkzeuge beschrieben und ein Überblick über mögliche Einsatzgebiete gegeben. Für das Verständnis der Umsetzung ist die Klärung der theoretischen Berechnungsgrundlagen erforderlich. Anschließend werden die möglichen Szenarien erarbeitet, woraus sich die Anforderungen an Funktionalität und Design für die Anwendung ergeben.\\ 
Die Konzipierung und Implementierung des exemplarischen Prototyps bilden den Kern dieser Arbeit, wobei dieser Prototyp wird in Architektur, Funktionalität und Design erläutert und schließlich in mehreren Testreihen evaluiert wird. 
\end{abstract}
\selectlanguage{english}
\begin{abstract}
This work includes the development of an application  which predicts the traffic light cycle for supporting a bike ride and gives a recommendation. In case of following it is supposed to enable passing green traffic lights without stopping.\\
Therefore it is analyzed which researches, projects and applications do already exist. It is going to be discussed whether for implementing a prototype a mobile application or rather a Adruinoinstallation is more convenient. Resting upon this decision the physical and technical base is explained. At this point definitions and developing tools are described and an overview of potential domains is given.\\
For comprehension the purification of theoretical basis of computation is necessary. Afterwards possible scenarios are worked out what from requirements of functionality and design result.\\
The conception and implementation of the showcase prototype are the core of this work. This is exemplified in architecture, functionality and design and finally evaluated in several test series.
\end{abstract}
\endgroup
