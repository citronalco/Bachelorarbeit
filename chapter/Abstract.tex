\begingroup
\let\titlepage\par
\let\endtitlepage
\let
\selectlanguage{ngerman}
\begin{abstract}
In dieser Arbeit wird eine Anwendung entwickelt, die zur Unterstützung der Fahrt mit dem Fahrrad Ampelphasen vorhersagt und eine Empfehlung ausspricht. Sofern diese eingehalten wird, soll ein Passieren der grünen Ampelphasen ohne Anhalten ermöglicht werden.\\
Hierzu wird analysiert welche Studien, Projekte oder Anwendungen es zu diesem Thema bereits gibt. Es wird diskutiert, ob sich für die Realisierung eines Prototyps eine mobile Anwendung oder eher eine \gls{Arduino}installation anbietet. Basierend auf den beiden möglichen Entscheidungswegen werden die technischen und physikalischen Grundlagen erklärt. Hier werden Definitionen und Entwicklungswerkzeuge beschrieben und ein Überblick über mögliche Einsatzgebiete gegeben. Für das Verständnis der Umsetzung ist die Klärung der theoretischen Berechnungsgrundlagen erforderlich. Anschließend werden die möglichen Szenarien erarbeitet, woraus sich die Anforderungen an Funktionalität und Design für die Anwendung ergeben.\\ 
Die Konzipierung und Implementierung des exemplarischen Prototyps bilden den Kern dieser Arbeit, wobei dieser Prototyp wird in Architektur, Funktionalität und Design erläutert und schließlich in mehreren Testreihen evaluiert wird. 
\end{abstract}
\selectlanguage{english}
\begin{abstract}
In this work, an application is developed that predicts to support the bike ride traffic signals and proposes a recommendation which, if adhered to, will enable smooth passage to green traffic lights.\\
First, it is analyzed that studies, projects or applications exist on this topic already. Then discusses whether offering a mobile application or rather a \gls{Arduino} installation for the realization of a prototype. Based on the decision of the technical and physical principles are explained. Here, then, definitions and development tools are described and an overview of possible applications. For understanding the implementation of clarifying the theoretical basis of calculation is required. Then, the possible scenarios are developed, resulting in the requirements of functionality and design for the application. \\
The design and implementation of the exemplary prototype form the core of this work. This is explained in architecture, functionality and design, and finally evaluated in several test series.
\end{abstract}
\endgroup
