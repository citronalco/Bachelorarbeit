\chapter{Anforderungsdefinition}
Dieses Kapitel soll helfen eine Vorstellung für die Anforderungen an die Ampelhinweissystem -App zu bekommen. Im heutigen High-Technologie Zeitalter spielt die Benutzbarkeit und Funktionalität des Produkts eine große Rolle. Es geht nicht nur darum, Interesse zu wecken, sondern auch die NutzerInnen für das Produkt zu begeistern. \\
Aus den oben genannten Szenarien werden im Folgenden die Anforderungen hergeleitet, die die zu entwickelnde \gls{Smartphone}-Anwendung erfüllen soll. Für die Umsetzung einer solchen Applikation ist die Positions- und Geschwindigkeitsbestimmung obligatorisch. Im Zusammenhang mit den Ampeldaten, bestehend aus Lage- und Schaltplan sollen die notwendigen Berechnungen durchgeführt und deren Ergebnisse als Empfehlung ausgesprochen.
\section{Funktionalität}
Die Bestimmung der Position, Fahrtrichtung und Geschwindigkeit des Fahrrads ist die zentrale Voraussetzung für eine zeitgerechte Empfehlungsaussprechung. Sie sollte ebenso schnell wie präzise erfolgen, sodass es keine Verzögerung der berechneten Ergebnisanzeige gibt und die Informationen eine hohe Genauigkeit betragen. \\
Von ebenso hoher Relevanz sind die Ampeldaten bestehend aus der genauen Position der \gls{LSA} und deren Schaltpläne. Es muss ermittelt werden, welche Ampel die am nächsten Passierte und daraus die Entfernung zu eben solcher. Die Prognose findet statt, sobald die Ampel circa 300 Meter voraus liegt. Bei einer größeren Entfernung von zum Beispiel einem Kilometer kann die Gefahr der Ablenkung bestehen. Sollte die Anwendung auf der Strecke durchgängig vorschlagen schneller zu fahren, könnte die Person dies als Ansporn sehen und so nicht mehr auf den Verkehr achten. Bei dieser doch recht kurzen Entfernung genügt für den zu entwickelnden Prototyp der Abstand als Berechnungsbestandteil in Luftlinie. Bei Optimierungsbedarf bietet es sich hierfrür an, Kartenmaterial einzubinden, um den genauen Abstand ermitteln zu können. Hierbei werden dann beeinflussende Elemente wie Hügel oder Kurven berücksichtigt.\\

\begin{itemize}
	\item Zugriff schnell (Datenbank)
	\item 
\end{itemize}

\section{Die graphische Oberfläche}
Ebenso wie die Bestimmung der Position und Geschwindigkeit spielt auch die graphische Darstellung eine große Rolle. Gerade bei dem Gebrauch im Verkehr ist die Eindeutigkeit und schnelle Erfassbarkeit der zu übermittelnden Informationen bedeutend.  \\
Um nicht von dem Verkehr abzulenken und diesen somit zu gefährden muss die Information, sowohl in Bedeutung als auch in Darstellung auf einen kurzen Blick erkennbar sein. Demnach sollte die Oberfläche möglichst einfach gehalten werden.\\
Da die Anwendung während der Fahrt nicht bedienbar sein muss -- denn auch das würde vom Verkehr ablenken -- besteht mehr Raum für die Informationsanzeige. 
Es ist davon auszugehen, dass die Anwendung ausschließlich im Freien Gebrauch findet. Dies und die dort herrschenden wechselnden Helligkeiten, zum Beispiel bei der Fahrt durch Schatten, setzen eine Verwendung von hohen Kontrasten voraus.
\subsubsection{Empfehlungsanzeige??? <-- Frau Görlitz fragen}
Im Allgemeinen sollte die Anwendung in der Lage sein die sich aus Kapitel \ref{chap:szenarien} resultierenden vier Systemzustände zu visualisieren und gegebenenfalls eine entsprechende Empfehlung auszusprechen.\\
Zur Verdeutlichung und einfacher Erfassbarkeit bietet es sich an, die entsprechenden Farben zu nehmen. Also ist für die Zustände \textit{b},\textit{c} und \textit{d}, die das Erreichen der Grünen Welle mit oder ohne Veränderung der eigenen Geschwindigkeit Grün und für den Zustand \textit{a}, der ausdrückt dass keine Weiterfahrt ohne Absteigen möglich ist, Rot zu verwenden.