\chapter{Die Anforderungsdefinition}
Dieses Kapitel soll durch Untersuchung helfen, Vorstellung für die Anforderungen an Ampelhinweissystem -App zu bekommen. Begriff Persona wird eingeführt, erklärt, entwickelt.
Mit Hilfe dieser Personas ... werden analysen gemacht + kritisch beurteilt.
Den Abschluss bildet das Ergebnis dieser Untersuchung in Zusammenfassung der wichtigsten Anforderungen an eine App.
\section{Personas}
\subsection*{Einleitung}
Im heutigen High-Technologie Zeitalter ist gerade die Benutzbarkeit bei der Entwicklung eines Produktes ein wichtiger Faktor, der von Software-Entwicklern beachtet werden muss. Die Anforderungen der Nutzer stehen dabei im Mittelpunkt. Es geht in erster Linie darum, jene zufrieden zu stellen und nicht nur Interesse, sondern auch Begeisterung beim potentiellen Kunden zu wecken. Verschiedene Methoden, diese Anforderungen besser zu identifizieren und erfüllen zu können, haben sich bereits verbreitet und basieren meistens auf einer präzisen Darstellung der Nutzer. Eine erprobte Methode hat der Software-Entwickler Alan Cooper eingeführt: Personas oder Archetypen von Nutzern.
\subsection*{Definition}
Fokus auf Gruppe spezifischer Nutzer bekommen blabla
\subsection*{Grund für Personas}
Effizienz mit Personas
\subsection*{Prototyp: Personas}
Um eine mögliche Anforderungsanalyse erarbeiten zu können, ist die
Wahl auf Personas, als Kriterium der Anforderung von Zielnutzern, gefallen. Auf den nachfol-
genden Seiten sind vier verschiedene Personas in einem übersichtlichen Tabellenprofil aufge-
listet.
\section{Funktionalität}
\section{Diagramme}
\subsection*{Sequenzdiagramm}
\subsection*{Ablaufdiagramm}
\subsection*{UML -- Diagramm}
\section{Die graphische Oberfläche}
