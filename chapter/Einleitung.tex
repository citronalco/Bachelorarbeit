\chapter{\label{chap:einleitung}Einführung}
\section{Motivation}
Im Berliner Verkehrswesen ist ein deutlicher Trend zu bemerken. Das Fahrrad wird zum ökologischen und gesundheitlichen, aktiven Lebensstil und wird dem hohen Verkehrsaufkommen der Automobile, insbesondere in der Stadtregion, entgegenwirken. “Fahrradfahren boomt in Berlin stärker als bislang bekannt”\cite{Mopo}\\\\
%Neue Fahrradwege und Vergrößerung des Fahrradstraßennetzes sind regionale Baumaßnahmen, die dabei aktuell diesen Fahrradtrend bekräftigen. \cite{Mopo} \\
Grund der neuen Fahrradeuphorie ist nicht zuletzt die erfolgreiche Etablierung der E-Bikes\footnote{ Elektrofahrrad. Ein Fahrrad mit elektrischem Hilfsmotor}. E-Bikes erfreuen sich großer Beliebtheit und ermöglichen auch längere Touren ohne große Anstrengung. (Vgl. \cite{ebikes} S.70ff)\\ 
Die Digitalisierung der Autoinnenräume mit Navigation und Bordelektronik sowie die Verbindungen zu \glspl{Smartphone} stellen aktuell keine Besonderheit mehr dar. Wird das Fahrrad nun als „vollwertiges“ Mitglied im Straßenverkehr angesehen, kann zusätzliche Elektronik wie Navigation die FahrradfahrerInnen unterstützen. Sicherheit und eine rechtzeitige Ankunft am Ziel sind die Hauptaspekte der VerkehrsteilnemerInnen. Das Halten an der Ampel kann dabei schnell zu Verzögerungen führen. Doch wer die Restzeit im Voraus kennt, kann sich darauf einstellen und so die verlorenen Zeitabschnitte reduzieren.
% ### ZIELSTELLUNG ###
\section{Zielstellung}
Der Fahrtfluss der RadfahrerInnen soll nicht unnötig unterbrochen werden. Rote Ampeln zwingen zum Anhalten -- das Anfahren kostet Kraft und ist deshalb unbeliebt. So kommt es, dass viele RadfahrerInnen die Straße bei rot überqueren und die Verkehrssicherheit aller gefährden, wo doch das das Radfahren an sich zur Gesundheit beiträgt und gut für die Umwelt ist. Angesichts des Nutzenpotentials eines Ampelinformationssystems lässt sich die Zielstellung klar und deutlich formulieren. Durch reibungsloses Passieren der Ampeln wird der Verkehr sicherer und das Radfahren attraktiver.\\
Um die Ampeldaten zu erfassen, gibt es verschiedene Möglichkeiten. Eine 100 prozentige Deckung erreicht man nicht einmal durch manuelle Ablesung jeder Ampel, denn viele \acrlongpl{LSA} werden verkehrsabhängig gesteuert. Fußgängerampeln beeinflussen erst nach Knopfdruck den Verkehr, Funkempfänger oder Infrarotdetektoren in Straßenbahnen oder Bussen wird durch Induktionsschleifen in der Fahrbahn der Verkehr erfasst und angepasst. Weiter sind Busse und Straßenbahnen in Berlin in der Lage, aus gewisser Entfernung über Funk Grün anzufordern, was ebenfalls in den Verkehr eingreift. ( Vgl. \cite{lsa_bln} S.4f) \\
%\textit{Die Berliner Verkehrslenkungszentrale stellt für diese Arbeit verkehrstechnische Unterlagen wie einen Lageplan, Signalzeitenpläne und Daten der verkehrsabhängigen Steuerung von \acrlongpl{LSA} von fünf Kreuzungen auf gewünschter Strecke als Basis für den zu entwickelnden Prototypen zur Verfügung.} 
Absolut entscheidend ist die Richtigkeit der Datengrundlage, also die Position und Signalschaltpläne der Ampeln, denn darauf baut die Funktionalität der Anwendung auf.
Für die Auswertung dieser Daten werden die potentiellen Wartezeiten an der nächsten Ampel vorzeitig errechnet und den FahrerInnen mitgeteilt. Resultierend können die NutzerInnen die Geschwindigkeit anpassen und die verbleibende Wegstrecke zur Ampel nutzen, um bei Grün ohne anzuhalten die Kreuzung zu überqueren. Für die Datenerhebung werden zugleich die mobilen Systeme der RadfahrerInnen genutzt. So kann zunächst der Prototyp des Ampelhinweissystems, beispielhaft für die Stadt Plau am See, entwickelt werden.\\\\
Das Ziel der Arbeit ist ein Konzept und dessen prototypische Anwendung eines Ampelhinweissystem, welches einem auf Basis der zu erfassenden Datengrundlage Informationen über die Ampelschaltung zukommen lässt und die NutzerInnen so interaktiv durch das Verkehrsnetz führt.
%
% AUFBAU DER ARBEIT
%
\section{Aufbau der Arbeit}
Zunächst wird analysiert welche Studien, Projekte oder Anwendungen es zu diesem Thema bereits gibt. Dann wird erläutert, ob sich für die Realisierung eines Prototyps eine mobile Anwendung oder eher eine \gls{Arduino}installation anbietet. Basierend auf der Entscheidung werden in Kapitel \ref{chap:grundlagen} die technischen und physikalischen Grundlagen erklärt. Hier werden also Definitionen und Entwicklungswerkzeuge beschrieben und ein Überblick über mögliche Einsatzgebiete gegeben. Für das Verständnis der Umsetzung ist die Klärung der theoretischen Berechnungsgrundlagen erforderlich. \\
Anschließend werden die möglichen Szenarien erarbeitet, woraus sich die Anforderungen an Funktionalität und Design für die Anwendung ergeben.\\ 
Die Kapitel \ref{chap:entwurf} und \ref{chap:implementierung} bilden mit der Konzipierung und Implementierung des exemplarischen Prototyps den Kern dieser Arbeit. Dieser wird in Architektur, Funktionalität und Design erläutert und schließlich in mehreren Testreihen evaluiert.\\ 
Den Abschluss dieser Arbeit bildet eine Zusammenfassung der Ereignisse dieser Arbeit und einen Auslick auf zukünftige Entwicklung hinsichtlich des Themas.
