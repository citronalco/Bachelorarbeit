\chapter{Einführung}
\section{Motivation}
Im Berliner Verkehrswesen ist ein deutlicher Trend zu bemerken. Das Fahrrad wird zum ökologischen und gesundheitlichen, aktiven Lebensstil und wird dem hohen Verkehrsaufkommen der Automobile, insbesondere in der Stadtregion, entgegenwirken. “Fahrradfahren boomt in Berlin stärker als bislang bekannt”\cite{Mopo}\\\\
Neue Fahrradwege und Vergrößerung des Fahrradstraßennetzes sind regionale Baumaßnahmen, die dabei aktuell diesen Fahrradtrend unterstützen. Grund der neuen Fahrradeuphorie ist nicht zuletzt die erfolgreiche Etablierung der E-Bikes\footnote{ Elektrofahrrad. Ein Fahrrad mit elektrischem Hilfsmotor}. 
E-Bikes erfreuen sich großer Beliebtheit und ermöglichen auch längere Touren ohne große Anstrengung.\\ 
Die Digitalisierung der Autoinnenräume mit Navigation und Bordelektronik sowie die Verbindungen zu Smartphones stellen aktuell keine Besonderheit mehr dar. Wird das Fahrrad nun als „vollwertiges“ Mitglied im Straßenverkehr angesehen, kann zusätzliche Elektronik wie Navigation und Blickmechanismen die FahrradfahrerInnen unterstützen.\\\\ 
Der Fahrtfluss des Radfahrers soll nicht unnötig unterbrochen werden. Dafür werden die potentiellen Wartezeiten an der nächsten Ampel vorzeitig errechnet und dem Fahrer mitgeteilt. Resultierend kann der Nutzer die Geschwindigkeit anpassen und die verbleibende Wegstrecke zur Ampel nutzen, um bei Grün ohne anzuhalten die Kreuzung zu überqueren. Für die Datenerhebung werden zugleich die mobilen Systeme der Radfahrer genutzt.
\section{Zielstellung}
-- nicht absteigen müssen \\
-- gut für Verkehr, Gesundheit \\
-- so wird weniger bei rot gefahren \\
-- Funktionalität / Wozu dient die Anwendung? \\
Um die Ampeldaten zu erfassen, gibt es verschiedene Möglichkeiten. Eine 100prozentige Deckung erreicht man nicht einmal durch manuelle Ablesung jeder Ampel, da circa 20 Prozent der \acrlongpl{LSA} in Berlin manuell gesteuert werden. Wenn man das mit Ampeln auf gegebener Teststrecke umsetzt, kann zunächst der Prototyp des Ampelhinweissystem entwickelt werden. \\
Die Berliner Verkehrslenkungszentrale stellt für diese Arbeit verkehrstechnische Unterlagen wie einen Lageplan, Signalzeitenpläne und Daten der verkehrsabhängigen Steuerung von \acrlongpl{LSA} von circa zehn Anlagen zur Verfügung.
Die Auswertung erfolgt entweder durch eine Smartphone-App oder durch ein \gls{LED}-Licht-System per Blinkfrequenz; beides am Lenker angebracht. Bei Nutzung des Telefons, nimmt man den integrierten \gls{GPS}-Sender, bei der zweiten Variante muss man das System mit einem ausstatten.
\\\\
Das Ziel der Arbeit ist ein Konzept und dessen prototypische Anwendung eines Ampelhinweissystem, welches einem auf Basis der zu erstellenden  Ampeldatenbank Informationen über die Ampelschaltung zukommen lässt und ihn so interaktiv durch das Verkehrsnetz führt.
\section{Aufbau der Arbeit}
Zunächst wird analysiert welche Studien, Projekte oder Anwendungen es zu diesem Thema bereits gibt. Auf Grundlage dessen wird entschieden, in welcher Art das Thema dieser Arbeit umgesetzt wird. Es wird erläutert, ob sich eine mobile Anwendung oder eher eine \gls{Arduino}installation anbietet. Anhand dessen werden im vierten Kapitel die technischen Grundlagen erklärt. Hier werden also Definitionen und Entwicklungswerkzeuge beschrieben und ein Überblick über mögliche Einsatzgebiete gegeben. Für das Verständnis der Umsetzung ist die Klärung der theoretischen Berechnungsgrundlagen erforderlich. \\
Im Anforderungsanalysekapitel werden die Anforderungen an Funktionalität und Usability Kriterien für die Anwendung ermittelt und strukturiert. \\
\textit{Es werden Personas, fiktive Benutzer, eingeführt und schließlich eine Zusammenfassung der herausgearbeiteten Anforderungen gestellt.}\\\\ 
Das sechste Kapitel bildet mit der Konzipierung den Kern dieser Arbeit. 
\textit{Architektur, Design, (Theorie --- funktioniert nicht wie in Konzeption beschrieben, weil... )
es wird auf.. einngegangen.... mobile Anwendung. app. anrduino. die nutzer auf ampeln hinweist und die dauer der phase. }
\textit{Zusammenfassend wird das Konzept am Ende von allen Personas nocheinmal kritisch betrachtet und evaluiert.}\\
Kapitel sieben beschreibt die Umsetzung des exemplarischen Prototyps. Dieser wird dann in Architektur, Funktionalität und Design erläutert und schließlich in mehreren Testreihen evaluiert. Anhand einiger Systemtests wird eine Optimierung der Anwendung herausgearbeitet und gegebenenfalls umgesetzt. 
\\\\Den Abschluss dieser Arbeit bildet eine Zusammenfassung der Ereignisse dieser Arbeit und einen Auslick auf zukünftige Entwicklung hinsichtlich des Themas.
