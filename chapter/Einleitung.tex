\chapter{\label{chap:einleitung}Einführung}
\section{Motivation}
Im Verkehrswesen ist ein deutlicher Trend zu bemerken, bei dem das Fahrradfahren Teil eines gesundheitsorientierten und aktiven Lebensstils ist und gleichzeitig dem hohen Verkehrsaufkommen der Automobile, insbesondere in der Stadtregion, entgegenwirkt. “Fahrradfahren boomt in Berlin stärker als bislang bekannt”. \cite{Mopo}\\\\
Als Bundeshauptstadt geht Berlin mit gutem Beispiel voran und plant, auf diesen Boom zu reagieren. Die Verkehrsstrategie des Senats sieht vor, dass die Fahrradnutzung bis zum Jahr 2025 20 Prozent des gesamten Verkehrs ausmachen soll. "Wir brauchen eine intelligente Konstruktion, die alle Verkehrsarten verbindet", sagte Berlins derzeitiger Bürgermeister Michael Müller (SPD). \cite{Mopo}\\
Ein Grund der neuen Fahrradeuphorie ist nicht zuletzt die erfolgreiche Etablierung der E-Bikes\footnote{ Elektrofahrrad. Ein Fahrrad mit elektrischem Hilfsmotor}. E-Bikes erfreuen sich großer Beliebtheit und ermöglichen auch längere Touren ohne große Anstrengung (vgl. \cite{ebikes} S.70ff).\\ 
Wird das Fahrrad nun als „vollwertiges“ Mitglied im Straßenverkehr angesehen, kann zusätzliche Elektronik wie Navigation die FahrradfahrerInnen unterstützen. Sicherheit und eine rechtzeitige Ankunft am Ziel sind die Hauptaspekte für die VerkehrsteilnehmerInnen. Das Halten an der Ampel kann dabei schnell zu Verzögerungen führen. Gerade für untrainierte FahrerInnen ist das Anhalten an roten Ampeln eine Frage von Kraft und Ermüdung, woraus Frustration und sinkende Akzeptanz für das Fahrrad als Verkehrsmittel folgen. Doch wer die Restzeit im Voraus kennt, kann sich darauf einstellen und sowohl die verlorenen Zeitabschnitte, als auch die Kraftanstrengung reduzieren. So wird nicht zuletzt die Attraktivität des Fahrrades als Verkehrsmittel weiter gesteigert. 
% ### ZIELSTELLUNG ###
\section{Zielstellung}
Der Fahrtfluss der RadfahrerInnen soll nicht unnötig unterbrochen werden. Rote Ampeln zwingen zum Anhalten -- das Anfahren kostet Kraft und ist deshalb unbeliebt. So kommt es, dass viele RadfahrerInnen die Straße bei Rot überqueren und hierdurch die Verkehrssicherheit aller gefährden. Durch reibungsloses Passieren der Ampeln wird das Radfahren durch die damit einhergehende Optimierung des Verkehrsflusses bzw. des Krafteinsatzes attraktiver und im Nebeneffekt auch noch sicherer.\\\\
Um die Ampeldaten zu erfassen, gibt es verschiedene Möglichkeiten. Eine hundertprozentige Deckung erreicht man nicht durch manuelle Ablesung aller Ampeln, denn viele \acrlongpl{LSA} werden verkehrsabhängig gesteuert. Fußgängerampeln beeinflussen erst nach Knopfdruck den Verkehr. Über Funkempfänger oder Infrarotdetektoren in Straßenbahnen oder Bussen wird durch Induktionsschleifen in der Fahrbahn der Verkehr erfasst und angepasst. Weiter sind Busse und Straßenbahnen in der Lage, aus gewisser Entfernung über Funk Grün anzufordern, was ebenfalls in den Verkehr eingreift (vgl. \cite{lsa_bln} S.4f). \\
%\textit{Die Berliner Verkehrslenkungszentrale stellt für diese Arbeit verkehrstechnische Unterlagen wie einen Lageplan, Signalzeitenpläne und Daten der verkehrsabhängigen Steuerung von \acrlongpl{LSA} von fünf Kreuzungen auf gewünschter Strecke als Basis für den zu entwickelnden Prototypen zur Verfügung.} 
Absolut entscheidend ist die Richtigkeit der Datengrundlage, also die Position und Signalschaltpläne der Ampeln, denn auf diese baut die Funktionalität der Anwendung auf.
Für die Auswertung dieser Daten werden die potentiellen Wartezeiten an der nächsten Ampel vorzeitig errechnet und den FahrerInnen mitgeteilt. Resultierend können die NutzerInnen ihre Geschwindigkeit anpassen und die verbleibende Wegstrecke zur Ampel nutzen, um bei Grün ohne anzuhalten die Kreuzung zu überqueren. Für die Datenerhebung werden zugleich die mobilen Systeme der RadfahrerInnen genutzt. So kann zunächst der Prototyp des Ampelhinweissystems, beispielhaft für die Stadt Plau am See, entwickelt werden.\\\\
Angesichts des Nutzenpotentials eines Ampelinformationssystems lässt sich die Zielstellung klar und deutlich formulieren. Das Ziel der Arbeit ist ein Konzept für ein Ampelhinweissystem und dessen prototypische Anwendung zu entwickeln, welches Informationen über die Ampelschaltung liefert und die NutzerInnen so interaktiv durch das Verkehrsnetz führt.
