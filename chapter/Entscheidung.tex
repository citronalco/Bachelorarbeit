\chapter{\label{chap:entscheidung}Lösungsansätze}
Zur Umsetzung der beschriebenen Ampelinformationsanwendung kommen zwei Möglichkeiten in die engere Wahl. In diesem Kapitel werden die Realisierung durch eine \gls{Smartphone}-\Gls{App} und die einer \gls{Arduino}-Anwendung gegenübergestellt. 
Arduino ist eine Open-Source-Elektronikplatt- form, basierend auf einfach bedienbarer Hard- und Software, die für interaktive Projekte vorgesehen ist. Mit eigener Entwicklungsumgebung und Programmiersprache lässt sich der Arduino steuern. Über viele Sensoren kann ein Arduino die Umgebung erfassen und beeinflusst die Umwelt mit \glspl{LED}, Motoren und anderen Akteuren. \cite{arduino_about}
Zu beachten sind die Komponenten Sensorik, Datenspeicherung, Stromversorgung und Darstellung der Informationen.\\
\begin{description}[leftmargin=0.7cm,style=nextline]
%GPS
  \item[\gls{GPS}] ~ Als Grundlage aller modernen Navigations- und Ortungssysteme im Bereich der Navigation ist \gls{GPS} für die Fahrradpositionsbestimmung obligatorisch. In einem \gls{Smartphone} ist ein \gls{GPS}-Empfänger inklusive, für eine \gls{Arduino}-Anwendung ein entsprechendes Modul vonnöten (vgl. \cite{arduino} S. 227).\\
 %Datenbankanbindung
 \item[Datenspeicherung] ~ Ein weiterer wichtiger Aspekt ist die Datenspeicherung. Die Position der Ampeln und die Phasen der Schaltpläne können sowohl in einer Datenbank als auch in einer Datei gespeichert und dort von der Anwendung angefragt und ausgewertet werden. Eine Client-Server Architektur aufbauen, um via Internet auf eine Datenbank zuzugreifen ist in beiden Fällen möglich. Alternativ dazu liefert Android die native Datenbank SQLite mit und bietet damit die Möglichkeit, Dateien im internen Speicher abzulegen und auszulesen.\\ 
Auch ein \gls{Arduino} kann Dank der SD Bibliothek auf eine SD-Karte zugreifen und enthaltene Dateien lesen oder schreiben. Für die Kommunikation zwischen Mikrocontroller und SD-Karte ist die SPI-Schnittstelle vonnöten, welche nur über Pins auf großen \gls{Arduino}-Boards -- und nicht auf den Kleinen, die ohne Weiteres in den Fahrradlenker passen -- verfügbar ist.\cite{arduino_sd}\\
  %WWW
 % \item[Internetverbindung] ~ Durch mobile Breitbandverbindung oder auch wahlweise per \gls{WLAN} ist beim \gls{Smartphone} eine Internet-Anbindung vorhanden. Das \gls{Arduino}-Board benötigt für die Netzwerkverbindung eine Erweiterung um das Ethernet-Shield. Dieses basiert auf dem Wiznet W5100 Ethernet-Chip, welcher einen Netzwerk-Stack mit UDP\footnote{ User Datagram Protocol: Netzwerkprotokoll für die verbindungslose Datenübertragung über das Internet } und TCP\footnote{ Transmission Control Protocol: Netzwerkprotokoll für bidirektionalen Datentransport} Unterstützung bietet (Vgl. \cite{arduino} S. 36).\\
%STROM
  \item[Stromversorgung] ~ Die Stromversorgung ist im \gls{Smartphone} durch den integrierten Akku gegeben. Die Laufzeit ist vom Typ abhängig, genügt jedoch für die alltägliche Radstrecke. Für die mobile Stromversorgung des \gls{Arduino}-Boards wird eine 9-Volt Batterie benötigt, die zusätzlichen Platz beansprucht und wassergeschützt und gut erreichbar angebracht werden muss. Es gibt weiterhin die Möglichkeit den Strom aus dem Nabendynamo zu gewinnen. Außerdem sollte dann der \gls{Arduino} in der Lage sein, Strom zu speichern, sodass die Anwendung beim Halt an der Ampel nicht ausschaltet.\\
  %DARSTELLUNG
  \item[Darstellung] ~ Ein Darstellungskonzept muss bei beiden Möglichkeiten erstellt werden. Auf dem \gls{Smartphone} ist besonders auf Erkennbarkeit bei schlechten Witterungsbedingungen, das ggf. spiegelnde Display berücksichtigend, zu achten. Dafür sind aufgrund der Displaygröße wesentlich mehr Informationen darstellbar. Als \gls{Arduino}-Anwendung sind lediglich ein paar helle \glspl{LED} am bzw. im Lenker erforderlich. Diese müssen jedoch eindeutig und intuitiv lesbar sein, um den vollen Informationsumfang zu gewähreisten. Da die darzustellenden Informationen nicht von hoher Komplexität sind, bietet sich hier der Arduino zur Umsetzung der Anwendung an. \\
\end{description}
\subsection*{Ergebnis}  
Die Entscheidung fällt auf die \gls{Smartphone}-Anwendung. Dort sind die benötigten Komponenten bereits integriert und müssen nicht bautechnisch erweitert werden. Auch die Erweiterungen die es bereits für das Fahrrad in Form einer mobilen Anwendung gibt, zeigen, dass Lösungen für oben genannte Probleme wie zum Beispiel die Nutzung bei schlechten Witterungsbedingungen existieren.
