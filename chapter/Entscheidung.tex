\chapter{Lösungsansätze}
Zur Umsetzung der beschriebenen Ampelinformationsanwendung kommen zwei Möglichkeiten in die engere Wahl. In diesem Kapitel werden die Realisierung durch eine \gls{Smartphone} \Gls{App} und die einer \gls{Arduino}-Anwendung gegenübergestellt. Zu beachten sind die Komponenten wie Sensorik, Internetverbindung, Stromversorgung und Darstellung der Informationen.\\
\begin{description}[leftmargin=0.7cm,style=nextline]
%GPS
  \item[\gls{GPS}] ~ Als Grundlage aller modernen Navigations- und Ortungssysteme im Bereich der Navigation ist \gls{GPS} für die Fahrradpositionsbestimmung obligatorisch. In einem \gls{Smartphone} ist ein \gls{GPS} Sensor inklusive, für eine \gls{Arduino}-Anwendung ein entsprechendes Modul vonnöten\footnote{ Vgl. \cite{arduino} S. 227}.\\
  % G-SENSOR
  \item[Beschleunigungssensor] ~ Der im \gls{Smartphone} integrierte Beschleunigungssensor ist ein Hardwaresensor der neben Neigung, Erschütterung, Vibration die Beschleunigung des Gerätes misst. Gerade bei Fahrten durch Tunnel ist dieser Sensor wichtig, da die Geschwindigkeitsberechnung mittels \gls{GPS}-Werten dort nicht möglich ist. Verdschiedene Beschleunigungssensoren gibt es auch für den \gls{Arduino}. Eine Platine deren Werte man nach Anklemmung auslesen kann\footnote{\url{http://bildr.org/2011/04/sensing-orientation-with-the-adxl335-arduino/}}.
\\
  %WWW
  \item[Internetverbindung] ~ Durch mobile Breitbandverbindung oder auch wahlweise per \gls{WLAN} ist beim \gls{Smartphone} eine Internet-Anbindung vorhanden. 
  Das \gls{Arduino}-Board benötigt vür die Internetverbindung eine Erweiterung um das Ethernet-Shield\footnote{ Vgl. \cite{arduino} S. 36}.\\
  %STROM
  \item[Stromversorgung] ~ Die Stromversorgung ist im \gls{Smartphone} durch den integrierten Akku gegeben. Die Laufzeit ist vom Typ abhängig, genügt jedoch für die alltägliche Radstrecke. Für die mobile Stromversorgung des \gls{Arduino}-Boards wird eine 9-Volt Batterie benötigt, die zusätzlichen Platz beansprucht und wassergeschützt und gut erreichbar angebracht werden muss. Sicher gibt es weiterhin die Möglichkeit den Strom aus dem Nabendynamo zu gewinnen -- dies bedarf jedoch zusätzlicher Arbeit. Außerdem sollte dann der \gls{Arduino} in der Lage sein Strom zu speichern, sodass die Anwendung beim Halt an der Ampel nicht ausschaltet.\\
  %DARSTELLUNG
  \item[Darstellung] ~ Ein Darstellungskonzept muss bei beiden Möglichkeiten erstellt werden. Auf dem \gls{Smartphone} ist besonders auf Erkennbarkeit bei schlechten Witterungsbedingungen, das ggf. spiegelnde Display berücksichtigend zu achten. Dafür sind aufgrund des vorhandenen Displays in gewisser Größe wesentlich mehr Informationen darstellbar. Als \gls{Arduino}-Anwendung sind lediglich ein paar helle \glspl{LED} am bzw. im Lenker erforderlich. Diese müssen doch eindeutig und intuitiv lesbar sein, um den vollen Informationsumfang zu gewähreisten.\\
 %Datenbankanbindung
 \item[Datenbankanbindung] ~ Ein weiterer wichtiger Aspekt ist die Datenbankanbindung. Mindestens die Position der Ampeln und die Phasen der Schaltpläne werden in einer Datenbankbank gespeichert und dort von der Anwendung angefragt und ausgewertet. Während man für die \gls{Arduino}-Anwendung eine Client-Server Architektur aufbauen und via Internet auf die Datenbank zugreifen muss, liefert Android die Datenbank SQLite mit.\\
\end{description}
\subsection*{Ergebnis}  
Auch wenn das Ampelinformationssystem als \gls{Arduino}-Anwendung minimalistischer ist sprechen die anderen Punkte dagegen. Beginnend bei der einfachen Stromversorgung bis hin zur vorhandenen Sensorik und internen Datenbank steht das \gls{Smartphone} weiter vorn und überzeugt durch seine Einfachheit in der Anwendung. Auch die vielen Erweiterungen die es bereits für das Fahrrad in Form einer mobilen Anwendung gibt, zeigen dass Lösungen für oben genannte Probleme wie die Nutzung bei schlechten Witterungsbedingungen existieren. Der Prototyp wird als Android Anwendung umgesetzt.
