\chapter{Konzeption}
% ### Achitektur ###
\section{Architektur}
Klassenstruktur..
\subsection{Technologien}
Für die Entwicklung des Prototyps fiel die Wahl der zu verwendeten Technologien zunächst auf
das AngularJS Framework, da bereits Vertrautheit mit Web Technologien vorlag.
\subsection{Navigation (bei App)}
% ### Design ###
\section{Das Design}
\subsection{Anzeigeelemente}
\subsubsection{Geschwindigkeit}
\subsubsection{Ampeln}
\subsubsection{Informationen}
\section{Ampeldatenanfrage und Auswertung}
\subsection{Sensoren}
\subsubsection{GPS} wird gebraucht für...
\section{Theorie}
Um die korrekte Umsetzung des Prototyps zu ermöglichen, müssen zunächst einmal prinzipielle
Theorien und Hintergründe diesen betreffend betrachtet werden.
grundlegendes Wissen über geographische Koordinaten sowie mathematische Voraussetzungen im Umgang mit diesen, müssen zur Ideenverwirklichung berücksichtigt werden.
\subsection{Die Berechnung der Entferung}
\subsection{Die Berechnung der Ankunft in Abhängigkeit der Geschwindigkeit}
\subsection{Die Berechnung der Dauer der Ampelphase?}
% ### Funktionalität ###
\section{Funktionalitäten --> Anforderungsanalyse?}
\subsection*{Sensorik (GPS)}
\subsection*{Open Street Map}
