\chapter{\label{chap:fazit}Ergebnis und Ausblick}
\section{Ampelhinweissystem}
Alle LSA aufnehmen = sehr zeitaufwändig, da Positionsdaten als Lageplan im \gls{PDF}-Format vorliegen. Keine genauen Latitude / longitude Werte vorhanden. An Ampeln in OSM kann man sich schlecht orientieren, da dort nur KraftfahrzeugAmpeln aufgeführt sind. Für Fahrräder ist zB die Abbiegeampel auf der gegenüberliegenden Straßenseite. \\
Kompliziert den Überblick übe die \gls{LSA}-Daten zu gewinnen. Pro Tag unterschiedliche Zeitabschnitte, manchmal mit bis zu zusätzlichen zwei Alternativschaltplänen. Dazu kommen Einschaltpläne plus Phasenübergänge. Weitere Daten wie Feindlichkeitsmatritzen usw.
\subsection{Datengrundlage}
Wie liegen die Daten vor? Echtzeitdaten benutzen! 
Schnittstelle zur Verfügung stellen - gerade für die verkehrsabhängigen Ampeln. Dann muss der Teil der Anwendung, welcher die Daten einliest, anders aufgebaut werden.

\section{Ausblick}

Alternative: Straßendaten, routennavigation\\ vorteile, ampel ist sicher, nachteile:
App wird viel größer (mehrere MB),  interaktion einbauen, navigation einbauen etc. 

\textit{Mit Straßendaten = andere Formel notwendig. Winkel / Hügel mit einbeziehen + Erdkugel etc.}\\

Unterschiede Klein-Großstadt? --> Datenbasis ist anders\\
