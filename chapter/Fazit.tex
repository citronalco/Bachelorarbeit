\chapter{\label{chap:fazit}Zusammenfassung und Ausblick}
Das Thema der Grünen Welle ist sehr interessant und wurde in den letzten Jahren oft diskutiert. Auch in Zukunft wird es mehr Untersuchungen zu diesem Thema geben, da eine Geschwindigkeitsempfehlung für alle Verkehrsteilnehmer viel Potential bietet. Intelligente Erweiterungen für Fahrräder erfreuen sich auch immer größerer Beliebtheit, werden kreativer und intelligenter.\\\\
Der im Rahmen dieser Arbeit entworfene Prototyp vereinigt diese beiden Themen in einer \gls{Smartphone}-\gls{App} für FahrradfahrerInnen. Die Anwendung versucht durch die Vermeidung von Halten an roten Ampeln, wenn es eigentlich nicht erforderlich sein müsste, eine entspannte Fahrt zu ermöglichen. Dafür werden manuell erfasste Ampeldaten, bestehend aus Position, Sigalzustände und geltender Zeitabschnitte, von der Anwendung eingelesen und ausgewertet. Anhand dieser Daten errechnet die Anwendung die optimale Geschwindigkeit, die eingehalten werden muss, um die nächstgelegene Ampel bei Grün zu passieren. Die Ergebnisse werden auf dem Display in einer intuitiven Form visuell umgesetzt.\\\\ 
Die Evaluierung wurde in zwei Testphasen unterteilt. Zuerst wurde der Algorithmus zur Bestimmung der nächsten Ampel überprüft und dann die Berechnung der Geschwindigkeitsempfehlung geprüft. Beide Testphasen konnten erfolgreich abgeschlossen werden. Bei der Geschwindigkeitsempfehlung wie auch bei der Ermittlung der nächsten Ampel gab es kleine Abweichungen, welche auf die Ungenauigkeit des \gls{GPS}-Sensors zurückzuführen sind. Abschließend lässt sich sagen, dass die Anwendung stabil und zuverlässig läuft.\\\\
%
% Probleme und Verbesserungen
%
Das System ist bei Bedarf auf verschiedene Weisen erweiterbar. So wäre eine Möglichkeit die der Aufnahme von personenbezogenen Daten wie die gewünschte Höchstgeschwindigkeit oder die maximale Beschleunigung.\\\\
Die Erstellung der Datengrundlage erwies sich trotz der überschaubaren Anzahl an Ampeln als sehr zeitaufwändig. Für eine Ausdehnung der geltenden Reichweite der Anwendung ist es vorteilhaft, andere Möglichkeiten zu finden.\\ 
In Zusammenarbeit mit den zuständigen Verkehrsleitzentralen könnte zum Beispiel eine Schnittstelle bereitgestellt werden, welche die Phasendaten in Echtzeit übermittelt, sofern es in deren Möglichkeiten liegt. So wäre eine Vorhersage selbst bei verkehrsabhängigen Ampeln möglich. Hierbei ist der Teil der Anwendung, welcher die Daten einliest, entsprechend anzupassen.\\
Eine weitere Option ist die Kommunikation zwischen \gls{Smartphone} und Ampel. Dazu ist eine Umrüstung der meisten Ampeln erforderlich, was eine gewisse Zeit in Anspruch nehmen wird. \cite{smart_lsa}\\\\
Bei einer Erweiterung um mehrere Gebiete ist ebenfalls über eine andere Datengrundlage nachzudenken. Bei einer großen Menge an Ampeln bietet sich eine Datenbank zur Speicherung an. Da die Applikation auch auf Geodaten basiert, wäre die Verwendung einer räumlichen Datenbank für Geographische Informationssysteme vorteilhaft. \\\\
Eine Integration der Routennavigation hätte ebenfalls gewisse Vorzüge. Hierbei wäre die nächste Ampel zuverlässig zu ermitteln und deren Signalwerte können eher ausgewertet werden. So werden auch nur die relevanten Abbiegeampeln berücksichtigt. Ohne Navigation ist es schwierig einen Abbiegewunsch vorherzusagen. In Zukunft könnte also die entwickelte Anwendung mit einer Navigationssoftware verknüpft werden.\\\\
Die Genauigkeit der \gls{GPS}-Daten war höher als erwartet. Bei einer Ausdehnung auf dichter besiedelte Gebiete und Großstädte werden wahrscheinlich andere Genauigkeitsergebnisse wegen der vielen und hohen Gebäude erzielt. Doch \gls{GPS}-Sensoren im \gls{Smartphone} werden zukünftig mehr Verbesserungen erhalten. 
Ein großer Fortschritt in Richtung Positionsbestimmung kann mit Galileo, dem europäischen \gls{GNSS} gemacht werden. In Kombination mit Daten von \gls{GPS} und \gls{GLONASS} könnten Ungenauigkeiten bald der Vergangenheit angehören. \cite{gnss}\\\\
Bei einer Ausweitung sind auch andere Faktoren zu beachten. So werden in ländlichen Gebieten viele Ampeln in der Nacht oder an Sonntagen ausgeschaltet -- nachts in Berlin nur 30 Prozent der Ampeln (Vgl. \cite{lsa_bln} S. 4). Auch befinden sich in großen Städten mehr verkehrsabhängige \glspl{LSA}, die neben den Einflüssen von FußgängerInnen auch auf öffentliche Verkehrsmittel reagieren und den Verkehr beeinflussen. Auch die Kreuzungen sind teilweise unübersichtlicher und die nächste Ampel schwerer zu ermitteln.\\\\
Schlussendlich lässt sich sagen, dass... blablasülz
