\chapter{Grundlagen}
Dieses Kapitel befasst sich mit sowohl den mathematischen als auch den technischen Grundlagen der zu behandelnden Thematik, welche für das weitere Verständnis der Arbeit beitragen. \textit{Auf mobile Anwendungen geht dieses Grundlagenkapitel nicht in besonderer Tiefe ein, da ein allgemeines Verständnis und Vertrautheit mit Konzepten und Technologien mobiler Anwendungen vorausgesetzt wird.}
\section{Mathematische Grundlagen}
\subsection*{Berechnung der Geschwindigkeitsempfehlung}
Präsentiert das System während der Anwendung eine Geschwindigkeitsempfehlung, ist diese abhängig von der Fahrtgeschwindigkeit und vom Abstand zur Ampel. Angenommen die Progressionsgeschwindigkeit $V$ wird zum Zeitpunkt $t_{1}$ ausgesprochen, die \gls {LSA} schaltet zum Zeitpunt $t_{2}$ auf Rot und Abstand zur Ampel beträgt $S$, dann gilt: 
\[ V = \frac{S}{t_{2} - t_{1}} \]
Um die ensprechende \gls{LSA} während der Grünphase zu passieren, muss letztendlich die empfohlene Geschwindigkeit $V$ eingehalten werden. Folgende Fälle werden beachtet:
\paragraph{Anhalten ist unvermeidbar:} Die Ampelschaltung erlaubt momentan kein reibungsloses Passieren. 
\textit{Umgesetzte Anzeigevariante: \textbf{rot}}
\paragraph{Konstante Weiterfahrt möglich:} Ist die empfohlene Geschwindigkeit gleich der aktuellen, ist ein reibungsloses Passieren bei beibehaltenem Tempo möglich. Es besteht kein Handlungsbedarf. 
\textit{Umgesetzte Anzeigevariante: \textbf{grün}}
\paragraph{Reibungsloses Passieren durch Beschleunigung möglich:} Zeigt die Ampel im Moment Grün und ist die empfohlene Geschwindigkeit höher als die aktuelle, ist ein reibungsloses Passieren durch Beschleunigung zu erreichen. Bei der Anzeige der Progressionsgeschwindigkeit ist selbstverständlich die geltende Höchstgeschwindigkeitsbegrenzung zu beachten. 
\textit{Umgesetzte Anzeigevariante: \textbf{grün}}
\paragraph{Reibungsloses Passieren duch Verlangsamen möglich:} Zeigt die Ampel im Moment Rot und ist die empfohlene Geschwindigkeit niedriger als die aktuelle, ist ein reibungsloses Passieren durch Verlangsamung zu erreichen.
\textit{Umgesetzte Anzeigevariante: \textbf{grün}}
%\subsection{Berechnung der Restrotanzeige}
\section{Technische Grundlagen}
\subsection{Arduino / Android-App}
\subsubsection{Mobile Sensing}
\textit{Der Beschleunigungssensor ist ein Hardwaresensor, der dazu benutzt wird, Position, Bewegung, Neigung, Erschütterung, Vibration und natürlich Beschleunigung des Gerätes zu messen.Es gibt bis zu 3-Achsen Beschleunigungssensoren, die meistens zum Erkennen der Ausrichtung des Smartphones genutzt werden und somit das Display beim Anschauen von Bildern, Webbrowsern oder Musikplayern in die passende Richtung vom Portrait-Modus (senkrecht) zum Landscape-Modus (waagrecht) zu drehen. In Kombination mit \gls{GPS} kann das Smartphone dank ihm sogar erkennen, welche Art Transportmittel (Fahrrad, Bus, U-Bahn) der Nutzer gerade benutzt und bestimmte Muster wie z.B. Rennen, Gehen oder Stehen unterscheiden.\\
\gls{GPS} erlaubt dem Smartphone sich selber zu lokalisieren und den exakten Standpunkt auf der Erde zu bestimmen. Es hilft locationbased\footnote{ ortsgebunden} Apps wie z.B Navigation, lokale Suche nach Shops, Restaurants etc. oder soziale Netzwerke wie Facebook oder Foursquare nötige Informationen zu ermitteln. Der Kompass erweitert die Möglichkeiten der Lokalisationsermittlung eines Smartphones. Er bestimmt den Winkel des Geräts relativ zum Nordpol der Erde. Der Kompass besitzt einen Magnet, der mit dem magnetischen Feld der Erde interagiert und sich entsprechend zu einem der Pole ausrichtet. Zusammen mit dem Gyroskop Sensor verbessern \gls{GPS} und Kompass die Präzision von locationbased Applikationen.Der Gyroskop Sensor bestimmt die Rotations- und Drehgeschwindigkeit des Smartphones auf seinen drei Achsen gegenüber dem Weltkoordinatensystem.}
\subsection{Backend mit nodejs / socket.io und MongoDB}
\subsection{Open-Street-Map}
