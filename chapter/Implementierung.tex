\chapter{\label{chap:implementierung}Der Prototyp}
Prototyp zeit, wie mittels GPS ... realisiert werden kann.
Design und Funktionalitäten werden ebenfalls vorgestellt
\section{Die Manifestdatei}
Das Android-Manifest dient der Festlegung wichtiger Eigenschaften der Anwendung und gehört zu jedem Android-Projekt. Die \gls{XML}-Datei (\texttt{AndroidManifest.xml}) ist im Hauptverzeichnis des Projekts zu finden und legt folgende Parameter fest: \\
Den Paketnamen des Programms:
\lstinputlisting[language=XML, firstline=3, lastline=3]{code/manifest.xml}
Für welche Android Versionen die Anwendung geschrieben wurde (\texttt{targetSdkVersion}) und das minimale \gls{API}-Level der Anwendung, also unter welcher Version die App noch ausgeführt werden kann:
\lstinputlisting[language=XML, firstline=4, lastline=6]{code/manifest.xml}
Die Berechtigung des \gls{GPS}-Zugriffs der Applikation, um Standortdaten des Endgeräts zu beziehen:
\lstinputlisting[language=XML, firstline=7, lastline=8]{code/manifest.xml}
Im \texttt{application}-Tag werden Variablen gesetzt, die das in der ActionBar dargestellte Icon und den Namen der Anwendung definieren. \textit{android:theme? + android:allowBackup?}
\lstinputlisting[language=XML, firstline=10, lastline=13]{code/manifest.xml}
Die \glspl{Activity} der Anwendung:
\lstinputlisting[language=XML, firstline=14, lastline=21]{code/manifest.xml}
Hier wird der Name der \gls{Activity}-Klasse gesetzt und im \texttt{intent-filter}-Tag festgelegt, dass diese \gls{Activity} als \texttt{MainActivity} beim Start der Anwendung ausgeführt wird. \textit{label:android:?}
\section{Verwendete Bibliotheken?}
\subsection{googlebla}
\section{MainActivity-Klasse}
\section{Umsetzung Szenarien}
\section{GeoLokalisierung}
%
% TEST
%
\chapter{Evaluierung}
\section{Systemtest}
ist das GPS schnell/genau genug fürs Radfahren?
Optimierung ggf. umsetzen\\
Personenbezogene Daten aufnehmen? Höchstgeschwindigkeit, maximale Beschleunigung
