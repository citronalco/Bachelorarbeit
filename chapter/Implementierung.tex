\chapter{\label{chap:implementierung}Der Prototyp}
In diesem Kaptitel wird die nach dem in Kapitel \ref{chap:entwurf} präsentierten Lösungswegs die detaillierte Beschreibung der technischen Realisierung der Anwendung vorgestellt.\\
Nach der Erklärung der Konfigurationsdateien wird auf die Umsetzung der Szenarien eingegangen. Im Zuge dessen werden die implementierten Algorithmen vorgestellt, wobei sich der erste mit dem Auffinden der nächsten relevanten Ampel befasst und der zweite die empfohlene Geschwindigkeit berechnet.
%
% MANIFEST
%
\section{Die Manifest- und build.gradle-Datei}
Das Android-Manifest dient der Festlegung wichtiger Eigenschaften der Anwendung und gehört zu jedem Android-Projekt. Die \gls{XML}-Datei (\texttt{AndroidManifest.xml}) ist im Hauptverzeichnis des Projekts zu finden und im Listing \ref{lst:manifest} abgebildet. In der zweiten Zeile wird hier der Paketname des Programms festgelegt. 
Im \texttt{application}-Tag werden Variablen gesetzt, die das in dargestellte Icon und den Namen der Anwendung definieren. Darüber hinaus wird hier die \gls{Activity} der Applikation definiert. Zuerst wird der Name der \gls{Activity} gesetzt. Die Variable \texttt{screenOrientation} legt das Format der Anzeige fest und verhindert ein automatisches Drehen des Bildschirms. 
\begin{center}
\rule{35em}{0.5pt} \lstinputlisting[language=XML, firstline=2, lastline=21, caption={AndroidManifest.xml}, label=lst:manifest]{code/manifest.xml}
 \rule{35em}{0.5pt}
\end{center}
Im \texttt{intent-filter}-Tag dass diese Activity beim Start der App ausgeführt wird. Hätte die Anwendung über mehrere \glspl{Activity} implementiert, würden die anderen ebenfalls hier aufgeführt werden.
Unterhalb des \texttt{application}-Tags, in Zeile 17, werden nun die Berechtigung des \gls{GPS}-Zugriffs der Applikation, um Standortdaten, also die jeweiligen geographischen Kordinaten des Endgeräts zu beziehen gesetzt.\\\\
Android Studio-Projekte enthalten eine Top-Level-Build-Datei und eine Build-Datei für jedes Modul. Die Build-Dateien heißen \texttt{build.gradle} und sind einfache Textdateien, die die Groovy\footnote{ Programmier- und Skriptsprache}-Syntax verwendend. Hier lässt sich der Build mit den Elementen, welche vom Android Plugin Gradle\footnote{ Gradle ist ein auf Java basirerndes Build-Management-Automatisierungs-Tool.} unterstütz werden, konfigurieren. \cite{android_build} \\ 
Listing \ref{lst:gradle} zeigt einen Ausschnitt aus der Build-Datei für die Anwendung. \\
Die Eigenschaft \texttt{compileSdkVersion} beschreibt mit welcher Android \gls{SDK} Version kompiliert werden soll und \texttt{buildToolsVersion} gibt an, welche Version des Build-Tools verwendet wird.
\begin{center}
\rule{35em}{0.5pt} \lstinputlisting[language=JSON, firstline=6, lastline=14,  caption={Auszug aus der build.gradle-Datei}, label=lst:gradle]{code/build.gradle} \rule{35em}{0.5pt}
\end{center}
Es folgt das \texttt{defaultConfig} Element, das die Kerneinstellungen und fügt diese dynamisch aus dem Build-System in die  Manifestdatei ein. Hier wird unter anderem festgelegt, für welche Android Versionen die Anwendung geschrieben wurde (\texttt{targetSdkVersion}) und das minimale \gls{API}-Level\footnote{ Eine Übersicht über die \gls{API}-Level und der dazugehörigen Android-Version findet sich unter \url{http://developer.android.com/guide/topics/manifest/uses-sdk-element.html} -- Zugriff: 03.03.2015} der Anwendung (\texttt{minSdkVersion}).\\ 
Letzteres ist auf "'10"' gesetzt und zeigt, dass die Anwendung auf Mobilgeräten auf denen mindestens die Android-Version 2.3.3 installiert sein muss. 
Die Android-Version wurde auf 2.3.3 festgelegt, da das die Mindestanforderung für das editieren der \texttt{SharedPreferences}, welche für das einmalige Anzeigen des Warndialogs benötigt werden, ist. Damit werden trotzdem noch 99,6\% der Geräte, die auf den Google-Play-Store zugreifen können, abgedeckt. \cite{android_version} 
%
% MainActivity
%
\clearpage
\section{MainActivity-Klasse}
Die Klasse \texttt{MainActivity} ist die \gls{Activity}, die beim Start der Anwendung ausgeführt wird.  Hier werden die Anzeigeelemente initialisiert und die zentralen Funktionalitäten delegiert.\\
Die Methode \texttt{onCreate} ist die erste Methode im \gls{Activity}-Lebenszyklus\footnote{ \url{http://developer.android.com/training/basics/activity-lifecycle/starting.html} -- Zugriff: 02.03.2015} und wird direkt nach dem Start der \gls{Activity} ausgeführt. Hier werden die Anzeigeelemente gesetzt und nach jeder Installation ein Dialog erstellt, der auf die oberste Priorität der Straßenverkehrsordnung hinweist.\\
Es wird eine Instanz des \texttt{JSONParser}-Objekts erstellt, welche die \gls{JSON}-Datei aus dem Ressourcenordner liest und in ein Objektarray umwandelt. Neben der \texttt{SpeedHandler}-Objektinstanz, welche für die Geschwindigkeitsempfehlung und dessen Anzeige zuständig ist, wird die des \texttt{GPSTracker} erstellt. Auf diesen ist der \texttt{OnSetListener} registriert, welcher ein Ereignis wirft sobald er die nächstgelegende Ampel gesetzt hat. Sobald dieser Fall eingetreten ist, wird der \texttt{SpeedHandler} beauftragt, den dazugehörigen Signalschaltplan zu holen, um dann anhand dieser Daten die Berechnungen durchzuführen und deren Ergebnisse anzuzeigen.   
%
% Umsetzung Szenarien
%
\section{Umsetzung Szenarien}
Wie in Kapitel \ref{chap:szenarien} beschrieben, lassen sich die sieben möglichen Szenarien auf insgesamt fünf komprimieren, da die Szenarien R2 und G2, genauso wie die Szenarien R3 und G3 dasselbe Ergebnis haben. Die zusammengefassten fünf Szenarien eignen sich für eine prototypische Umsetzung, welche im Folgenden beschrieben wird.
%
% Einlesen der Ampeldaten
%
\subsection{Einlesen der Ampeldaten}
Neben der Geräteposition bilden die Ampelposition und deren Signalschaltpläne die Grundlage der Anwendung. Hierbei ist das korrekte Einlesen und Auswerten der Daten obligatorisch. Die Aufgabe des \texttt{JSONParsers} ist es also, die manuell erstellte \gls{JSON}-Datei, welche die Ampeldaten beinhaltet, richtig zu konvertieren. Die Klasse \texttt{JSONParser} liest also die Datei ein und wandelt dann die enthaltenen \gls{JSON}-Ampelobjekte in Java-Ampelobjekte um.\\ 
Hierfür wird das \gls{JSON}-Array durchlaufen und für jedes enthaltende Objekt ein Ampelobjekt erzeugt. Aus den Werten \texttt{lat} und \texttt{lon}, stehend für latitude und longitude, wird ein \texttt{Location}-Objekt erzeugt, das als Attribut gesetzt wird. Der boolsche Wert \texttt{dependsOnTraffic} ist auf \texttt{true} gesetzt, sobald die \gls{LSA} verkehrsabhängig ist. Ist dies der Fall, ist eine Vorhersage der Ampel aufgrund beeinflussender Parameter nicht möglich und es wird ein Objekt mit den beiden genannten Attributen und dem bezeichnenden Namensattribut erzeugt. Andernfalls werden die zu der Ampel hinzugehörigen Schaltpläne durchlaufen und für jeden ein selbiges Objekt erzeugt, welche in einem Array gespeichert dem Ampelobjekt als weiteres Attribut übergeben wird. Es ist zu beachten, dass jeder Signalschaltplan über ein Array mit Tagen verfügt. Denn ein Schaltplan kann an mehreren Tagen gültig sein. Die Tage sind als Strings gespeichert und werden für die weitere Verwendung in Integer umgewandelt, wobei die Woche am Sonntag beginnt. Der Sonntag wird also als \texttt{1} gespeichert, der Montag als \texttt{2} und so weiter. 
%
% nächste Ampel
%
\subsection{Ermittlung der nächsten Ampel}
Zur Ermittlung der nächsten Ampel ist zunächst die eigene Position vonnöten. Dazu stellt der in dem \texttt{android.location}-Paket enthaltene \texttt{LocationManager} die entsprechenden Schnittstellen bereit. Für den Empfang der \gls{GPS}-Daten implementiert die Klasse \texttt{GpsTracker} das Interface \texttt{LocationListener} und registriert sich beim \texttt{LocationManager} für \gls{GPS}-Updates.
Letzteres ist im Listing \ref{lst:gps1} abgebildet. Die Methode erwartet neben dem Providertyp und dem zu benachrichtigenden Listener (\texttt{this}) zwei weitere Parameter. Der erste zeigt an nach wie vielen Millisekunden ein Update gesendet werden soll, der zweite dass ein Update nur gesendet werden soll, wenn sich die Position um die angegebene Distanz verändert hat.
\begin{center}
\rule{35em}{0.5pt}
\lstinputlisting[language=Java, firstline=1, lastline=3, caption={Registrierung für GPS-Updates in GpsTracker.java}, label=lst:gps1]{code/gpstracker.java}
\rule{35em}{0.5pt}
\end{center}
Mit der Implementierung des \texttt{LocationListener} wird unter anderem die Methode \texttt{onLocationChanged}, welche aufgerufen wird, sobald neue Positionsdaten vorhanden sind implementiert. Das hier gespeicherte \texttt{Location}-Objekt liefert neben dem Längen- und Breitengrad auch die Genauigkeit des \gls{GPS}-Signals in Metern und Geschwindigkeit in Metern pro Sekunde. Sofern die Genauigkeit nicht zu gering ist, wird mit ihm wird die Methode \texttt{getNearestLSA} aufgerufen, welche die nächste Ampel bestimmt.\\
Die Methode \texttt{getNearestLSA} ermittelt zunächst alle Ampeln im festgelegten Radius und speichert diese zusammen mit der Distanz zum Gerät in einer temporären Liste. Dann werden diese wieder durchlaufen und die Distanzen verglichen. Die Ampel, zu der sich die Entfernung zum Gerät verringert und die kleinste Distanz zum Gerät hat, ist die Gesuchte. Ist die nächste Ampel ermittelt, kann der \texttt{OnSetListener} der \texttt{MainActivity} diese zusammen mit der Geräteposition übergeben. \\\\
Vergrößert sich die Distanz zu der \gls{LSA} wieder befindet sie sich außerhalb des durchsuchten Radius bedeutet das, dass der Fahrer oder die Fahrerin sich von der Ampel oder der Kreuzung entfernt. Also werden diese dann zusammen mit den anderen gespeicherten Ampeln gelöscht. 
\clearpage
\begin{center}
%\rule{35em}{0.5pt}
\lstinputlisting[language=Java, firstline=12, lastline=64, caption={getNearestLSA() GpsTracker.java}, label=lst:gps]{code/gpstracker.java}
\rule{35em}{0.5pt}
\end{center}
%
% v = s/t
%
\subsection{Berechnung der Geschwindigkeitsempfehlung}
An dieser Stelle wird die Berechnung der Geschwindigkeitsempfehlung erklärt und im Listing \ref{lst:speed} veranschaulicht.
Sobald alle benötigten Daten bestehend aus aktueller, eigener Position, Entfernung zur nächsten Ampel und deren Signalschaltplan wird die benötigte Geschwindigkeit errechnet, die Ampel bei Grün zu erreichen. Diese Geschwindigkeit wird mit er Formel: 
\[ v = \frac{s}{t_{2} - t_{1}} \]
berechnet, wobei $t_1$ die aktuelle Sekunde, $t_2$ der Zeitpunkt zu dem die Ampel auf Rot schaltet und $s$ die Entfernung zur Ampel ist. \\
Für FahrradfahrerInnen ist die Beschleunigung ebenfalls von nicht geringer Bedeutung, weil sie begrenzt ist. Die Formel für die Beschleunigung lautet:
\[ a = \frac{v}{(t_{2} - t_{1})^{2}} \] 
Die Variable $v$ ist hier die errechnete Empfehlungsgeschwindigkeit.\\\\
Anhand der Ergebnisse wird die \gls{GUI} aktualisiert. Hier wird berücksichtigt, ob die errechnete Geschwindigkeit, sowie die Beschleunigung innerhalb festgelegter Parameter liegen. Da bei einer Handlungsaufforderung zusätzlich zur Empfehlung der Countdown der Ampel dargestellt wird, geschieht dies sekündlich.\\
Jede \gls{App} hat ihren eigenen \gls{UI}-Thread und nur Objekte, die auf dem \gls{UI}-Thread ausgeführt werden haben Zugang zu anderen Objekten auf diesem Thread. Um Daten aus einem Hintergrundthread auf den \gls{UI}-Thread zu verschieben wird der \texttt{Handler} verwendet.
Also wird in der \texttt{SpeedHandler}-Klasse eine \texttt{Handler}- und eine \texttt{Runnable}-Instanz erzeugt. Ein \texttt{Handler} ist eine Schnittstelle für die Kommunikation zwischen Threads. Jede \texttt{Handler}-Instanz wird mit einem Thread und dessen Warteschlange verbunden. Ein \texttt{Runnable} ist ein Objekt, das etwas ausführen kann. Das \texttt{Runnable} wird nun mittels \texttt{post()} im \texttt{Handler} abgelegt, wodurch \texttt{Runnable} in der \texttt{MainActivity} aus dem \texttt{Handler} ausgelesen und gestartet wird. 
%
% Installationsanleitung
%
\section{Installationsanleitung}
Um die Anwendung zu installieren muss deren .apk-Datei auf das Gerät geladen und von dort gestartet werden. Die .apk-Datei ist das als Zwischencode ausführbare Kompilat, welches dann auf dem Gerät zu Plattformcode kompiliert wird. Zum Auffinden und ausführen der Datei wird ein Dateimanager, wie zum Beispiel der kostenlose ASTRO File Manager\footnote{ Der ASTRO File Manager steht im Google Play Store unter \url{https://play.google.com/store/apps/details?id=com.metago.astro&hl=de} bereit.}  benötigt. 
