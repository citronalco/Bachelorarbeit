\newglossaryentry{C2Xg}{
    name={C2X}, 
    description={ direkter Informationsaustausch zwischen Fahrzeugen jeglicher Art, Verkehrsleittechnik wie z.B. Lichtsignalanlagen und Verkehrsleitzentralen}
}
%%% define the acronym and use the see= option
\newglossaryentry{C2X}{type=\acronymtype, name={C2X}, description={Car-to-X oder Vehicle-to-X}, first={Car-to-X (C2X)\glsadd{C2Xg}}, see=[Glossary:]{C2Xg}}
\newglossaryentry{C2Ig}{
    name={C2I}, 
    description={ direkter, drahtloser Datenaustausch zwischen Fahrzeugen jeglicher Art und infrastrukturellen Einrichtungen wie Funkbaken und Lichtsignalanlagen auf Basis von \gls{WLAN}, Bluetooth oder \gls{DSRC}}
}
\newglossaryentry{C2I}{type=\acronymtype, name={C2I}, description={Car-to-Infrastructure oder Vehicle-to-Infrastructure}, first={Car-to-Infrastructure (C2I)\glsadd{C2Ig}}, see=[Glossary:]{C2Ig}
}
\newglossaryentry{DSRCg}{
    name={DSRC}, 
    description={ funkgestützte sicherheitsrelevante und private Dienste, die in der Automotive-Technik von mobilen Stationen ausgeführt werden können}}
\newglossaryentry{DSRC}{type=\acronymtype, name={DSRC}, description={Dedicated Short Range Communication}, first={Dedicated Short Range Communication (DSRC)\glsadd{DSRCg}}, see=[Glossary:]{DSRCg}
}
\newglossaryentry{Smartphone}{
    name={Smartphone}, 
    description={ Mobiltelefon, das sich von einem klassischen Mobiltelefon durch einen größeren berührungsempfindlichen Bildschirm (Touchscreen) und diverse Sensoren, wie dem \gls{GPS} unterscheidet. So ist eine Interaktion mit der Umgebung und den AnwenderInnen möglich}
}
\newglossaryentry{Arduino}{
    name={Arduino}, 
    description={ Die Arduino-Plattform ist eine quelloffene, aus Soft- und Hardware bestehende Physical-Computing-Plattform. Die Hardware besteht aus einem einfachen I/O-Board mit einem Mikrocontroller und analogen und digitalen Ein- und Ausgängen. Die Entwicklungsumgebung basiert auf Processing, die auch technisch weniger Versierten den Zugang zur Programmierung und zu Mikrocontrollern erleichtern soll}
}
\newglossaryentry{Activity}{
    name={Activity}, 
    description={ Eine Activity ist eine Android-Anwendungskomponente, welche die Bildschirm, beziehungsweise eine Ansicht zur Interaktion bereitstellt},
    plural={Activities}
}
\newacronym{GPS}{GPS}{Global Positioning System}

\newacronym{A-GPS}{A-GPS}{Assisted Global Positioning System}
\newacronym{GNSS}{GNSS}{Global Navigation Satellite System}
\newacronym{GLONASS}{GLONASS}{Global Navigation Satellite System; das russische \gls{GNSS}}
\newacronym{LBS}{LBS}{Location Based Service}
\newacronym{API}{API}{Application Programming Interface}
\newacronym{App}{App}{Applikation}
\newacronym{JSON}{JSON}{JavaScript Object Notation}
\newacronym{PDF}{PDF}{Portable Document Format}
\newacronym{WLAN}{WLAN}{Wireless Local Area Network}
\newacronym{UMTS}{UMTS}{Universal Mobile Telecommunications System}
\newacronym[longplural={Lichtsignalanlagen}]{LSA}{LSA}{Lichtsignalanlage}
\newacronym[longplural={Licht-emittierende Dioden}]{LED}{LED}{Licht-emittierende Diode}
\newacronym{ART}{ART}{Android Runtime}
\newacronym{SDK}{SDK}{Software Development Kit}
\newacronym{NDK}{NDK}{Native Development Kit}
\newacronym{HAL}{HAL}{Hardware Abstraction Layer}
\newacronym{VM}{VM}{Virtual Maschine}
\newacronym{JVM}{JVM}{Java Virtual Maschine}
\newacronym{DDMS}{DDMS}{Dalvik Debug Monitor Server}
\newacronym{XML}{XML}{Extensible Markup Language}
\newacronym{SQL}{SQL}{Structured Query Language}
\newacronym{UI}{UI}{User Interface}
\newacronym{GUI}{GUI}{Graphical User Interface}
