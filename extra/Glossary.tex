\newglossaryentry{C2Xg}{
    name={C2X}, 
    description={ direkter Informationsaustausch zwischen Fahrzeugen jeglicher Art, Verkehrsleittechnik wie z.B. Lichtsignalanlagen und Verkehrsleitzentralen}
}
%%% define the acronym and use the see= option
\newglossaryentry{C2X}{type=\acronymtype, name={C2X}, description={Car-to-X oder Vehicle-to-X}, first={Car-to-X (C2X)\glsadd{C2Xg}}, see=[Glossary:]{C2Xg}}
\newglossaryentry{C2Ig}{
    name={C2I}, 
    description={ direkter, drahtloser Datenaustausch zwischen Fahrzeugen jeglicher Art und infrastrukturellen Einrichtungen wie Funkbaken und Lichtsignalanlagen auf Basis von \gls{WLAN}, Bluetooth oder \gls{DSRC}}
}
\newglossaryentry{C2I}{type=\acronymtype, name={C2X}, description={Car-to-Infrastructure oder Vehicle-to-Infrastructure}, first={Car-to-Infrastructure (C2I)\glsadd{C2Ig}}, see=[Glossary:]{C2Ig}
}
\newglossaryentry{DSRCg}{
    name={DSRC}, 
    description={ funkgestützte sicherheitsrelevante und private Dienste, die in der Automotive-Technik von mobilen Stationen ausgeführt werden können}}
\newglossaryentry{DSRC}{type=\acronymtype, name={DSRC}, description={Dedicated Short Range Communication}, first={Dedicated Short Range Communication (DSRC)\glsadd{DSRCg}}, see=[Glossary:]{DSRCg}
}

\newglossaryentry{RESTg}{
    name={REST}, 
    description={beschreibt eine zustandslose Verbindung. Web-Dienste, welche alle nötigen Anforderungen besitzen, können als \gls{REST}ful beschrieben werden. \gls{REST} \glspl{API} stellen also eine einheitliche uniforme Schnittstelle dar. Durch die HTTP-Operationen \textsc{GET},\textsc{POST},\textsc{PUT} und \textsc{DELETE}, können alle CRUD-Methoden (Create, Read, Update, Delete) abgebildet werden und so beispielsweise Datenbankzugriffe gesteuert werden}}
\newglossaryentry{REST}{type=\acronymtype, name={REST}, description={REpresentational State Transfer}, first={\glsadd{RESTg}}, see=[Glossary:]{RESTg}}
\newglossaryentry{Smartphone}{
    name={Smartphone}, 
    description={ Mobiltelefon, das sich von einem klassischen Mobiltelefon durch einen größeren berührungsempfindlichen Bildschirm (Touchscreen) und diverse Sensoren, wie dem \gls{GPS} unterscheidet. So ist eine Interaktion mit der Umgebung und den AnwenderInnen möglich}
}
\newglossaryentry{Arduino}{
    name={Arduino}, 
    description={ Die Arduino-Plattform ist eine quelloffene, aus Soft- und Hardware bestehende Physical-Computing-Plattform. Die Hardware besteht aus einem einfachen I/O-Board mit einem Mikrocontroller und analogen und digitalen Ein- und Ausgängen. Die Entwicklungsumgebung basiert auf Processing, die auch technisch weniger Versierten den Zugang zur Programmierung und zu Mikrocontrollern erleichtern soll}
}
\newacronym{GPS}{GPS}{Global Positioning System}
\newacronym{API}{API}{Application Programming Interface}
\newacronym{App}{App}{Applikation}
\newacronym{WLAN}{WLAN}{Wireless Local Area Network}
\newacronym{UMTS}{UMTS}{Universal Mobile Telecommunications System}
\newacronym[longplural={Lichtsignalanlagen}]{LSA}{LSA}{Lichtsignalanlage}
\newacronym{MIT}{MIT}{Massachusetts Institute of Technology}
\newacronym{TUM}{TUM}{Technische Universität München}
\newacronym[longplural={Licht-emittierende Dioden}]{LED}{LED}{Licht-emittierende Diode}
\newacronym{GLOSA}{GLOSA}{Green Light Optimal Speed Advisory}
\newacronym{OSM}{OSM}{Open Street Map}
\newacronym{SQL}{SQL}{Structured Query Language}
\newacronym{JSON}{JSON}{JavaScript Object Notation}
\newacronym{NoSQL}{NoSQL}{Not only \gls{SQL}}
\newacronym{UI}{UI}{User Interface}
\newacronym{GUI}{GUI}{Graphical User Interface}
\newacronym{HTML}{HTML}{HyperText Markup Language}
\newacronym{CSS}{CSS}{Cascading Style Sheet}
\newacronym{XML}{XML}{Extensible Markup Language}
